\section{Безопасность жизнедеятельности}
\label{section:OSH}
В процессе жизнедеятельности человек подвергается воздействию различных опасностей, под которыми обычно понимают явления, процессы, объекты, способные в определённых условиях наносить ущерб здоровью человека непосредственно или косвенно, т.е. вызывать различные нежелательные последствия~\cite{MSTUCA_OT}.

Человек подвергается воздействию опасностей и в своей трудовой деятельности. Эта деятельность осуществляется в пространстве, называемом производственной средой. В условиях производства на человека в основном действуют техногенные, т.е. связанные с техникой, опасности, которые принято называть опасными и вредными производственными факторами~\cite{MSTUCA_OT}.

Опасным производственным фактором (ОПФ) называется такой производственный фактор, воздействие которого на работающего в определённых условиях приводит к травме или к другому внезапному резкому ухудшению здоровья. Травма~--- это повреждение тканей организма и нарушение его функций внешним воздействием. Травма является результатом несчастного случая на производстве, под которым понимают случай действия опасного производственного фактора на работающего при выполнении им трудовых обязанностей или заданий руководителя работ~\cite{MSTUCA_OT}.

Вредным производственным фактором (ВПФ) называется такой производственный фактор, воздействие которого на работающего в определённых условиях приводит к заболеванию или снижению трудоспособности. Заболевания, возникающие под действием вредных производственных факторов, называются профессиональными~\cite{MSTUCA_OT}.

К опасным производственным факторам следует отнести, например~\cite{MSTUCA_OT}:
\begin{itemize}
	\item электрический ток определённой силы;
	\item раскалённые предметы;
	\item возможность падения с высоты самого работающего либо различных деталей и
предметов;
	\item оборудование, работающее под давлением выше атмосферного, и т.д.
\end{itemize}
	
К вредным производственным факторам относятся~\cite{MSTUCA_OT}:
\begin{itemize}
	\item неблагоприятные метеорологические условия;
	\item запыленность и загазованность воздушной среды;
	\item воздействие шума, инфра- и ультразвука, вибрации; 
	\item наличие электромагнитных полей, лазерного и ионизирующих излучений и др.
\end{itemize}

Все опасные и вредные производственные факторы подразделяются на физические, химические, биологические и психофизиологические~\cite{MSTUCA_OT}.

К физическим факторам относят электрический ток, кинетическую энергию движущихся машин и оборудования или их частей, повышенное давление паров или газов в сосудах, недопустимые уровни шума, вибрации, инфра- и ультразвука, недостаточную освещенность, электромагнитные поля, ионизирующие излучения и др~\cite{MSTUCA_OT}.

Химические факторы представляют собой вредные для организма человека вещества в различных состояниях~\cite{MSTUCA_OT}.

Биологические факторы~--- это воздействия различных микроорганизмов, а также растений и животных~\cite{MSTUCA_OT}.

Психофизиологические факторы~--- это физические и эмоциональные перегрузки, умственное перенапряжение, монотонность труда~\cite{MSTUCA_OT}.

Чёткой границы между опасным и вредным производственными факторами часто не существует. Рассмотрим в качестве примера воздействие на работающего расплавленного металла. Если человек попадает под его непосредственное воздействие (термический ожог), это приводит к тяжёлой травме и может закончиться смертью пострадавшего. В этом случае воздействие расплавленного металла на работающего
является согласно определению опасным производственным фактором~\cite{MSTUCA_OT}.

Если же человек, постоянно работая с расплавленным металлом, находится под действием лучистой теплоты, излучаемой этим источником, то под влиянием облучения в организме происходят биохимические сдвиги, наступает нарушение деятельности сердечно-сосудистой и нервной систем. Кроме того, длительное воздействие инфракрасных лучей вредно влияет на органы зрения~--- приводит к помутнению хрусталика. Таким образом, во втором случае воздействие лучистой теплоты от расплавленного металла, на организм работающего является вредным производственным фактором~\cite{MSTUCA_OT}.

Состояние условий труда, при котором исключено воздействие на работающих опасных и вредных производственных факторов, называется безопасностью труда. Безопасность жизнедеятельности в условиях производства имеет и другое название — охрана труда. В настоящее время последний термин считается устаревшим~\cite{MSTUCA_OT}.

Охрана труда определялась как система законодательных актов, социальноэкономических, организационных, технических, гигиенических и лечебно-профилактических мероприятий и средств, обеспечивающих безопасность, сохранение здоровья и работоспособности в процессе труда~\cite{MSTUCA_OT}.

Одна из самых распространённых мер по предупреждению неблагоприятного воздействия на работающих опасных и вредных производственных факторов использование средств коллективной и индивидуальной защиты. Первые из них предназначены для одновременной защиты двух и более работающих, вторые~--- для защиты одного работающего. Так, при загрязнении пылью воздушной среды в процессе производства в качестве коллективного средства защиты может быть рекомендована общеобменная приточно-вытяжная вентиляция, а в качестве индивидуального респиратор~\cite{MSTUCA_OT}.

Шум является одним из наиболее важным и трудноустранимым вредным фактором. Рассмотрим физическую суть данного фактора, его воздействие на человека и способы защиты более подробно.

% **************************************************
\subsection{Шум и его характеристики}
Шумом принято называть любой нежелательный звук, воспринимаемый органом слуха человека. Шум представляет собой беспорядочное сочетание звуков различной интенсивности и частоты. Пространство, в котором распространяются звуковые волны, называется звуковым полем. Давление и скорость движения частиц воздуха в каждой точке звукового поля изменяются во времени. В результате колебаний, создаваемых источником звука, в воздухе возникает звуковое давление, которое накладывается на атмосферное~\cite{TechLib_NoiseCharacteristics}.

Зависимость звукового давления от времени можно представить в виде суммы конечного или бесконечного числа простых синусоидальных колебаний этой величины. Каждое такое колебание характеризуется своим среднеквадратичным значением физической величины и частотой. Частота звука характеризуется числом колебаний звуковой волны в единицу времени (секунду) и измеряется в герцах. По частоте звуковые колебания подразделяются на три диапазона: инфра-звуковые с частотой колебаний менее 20 Гц, звуковые - от 20 до 20 000 Гц и ультразвуковые - более 20 000 Гц. Органы слуха человека воспринимают звуковые колебания в интервале частот от 20 до 20 000 Гц. Звуковой диапазон принято подразделять: на низкочастотный~--- до 400 Гц, среднечастотный~--- от 400 до 1000 Гц и высокочастотный~--- свыше 1000 Гц~\cite{TechLib_NoiseCharacteristics}.

При распространении звуковой волны происходит перенос энергии. Энергия, переносимая звуковой волной в единицу времени через поверхность, перпендикулярную направлению распространения волны, называется интенсивностью звука $I$~\cite{TechLib_NoiseCharacteristics}.

Звуковое давление и интенсивность звука могут изменяться по величине в широких пределах: по давлению~--- до 108 раз, а по интенсивности~--- до 1016 раз. Важное значение имеет также то, что ухо человека реагирует не на абсолютное, а на относительное изменение интенсивности звука, поскольку интенсивность звука (ощущения человека при шуме) пропорциональна логарифму количества энергии раздражителя. Поэтому были введены логарифмические величины~--- уровни интенсивности и звукового давления, выражаемые в децибелах (дБ)~\cite{TechLib_NoiseCharacteristics}.

Уровень интенсивности звука~\cite{TechLib_NoiseCharacteristics}:
\begin{equation}
	L_I = 10 \lg{\left(\frac{I_I}{I_0}\right)},
\end{equation}
\begin{explanation}
	где &$I_I$& интенсивность звука в данной точке, $\text{Вт}/\text{м}^2$; \\
	&$I_0$& интенсивность звука, соответствующая порогу слышимости, 10--12 $\text{Вт}/\text{м}^2$ на частоте 1000 Гц.
\end{explanation}

Уровень звукового давления определяется как~\cite{TechLib_NoiseCharacteristics}:
\begin{equation}
	L_p = 20\lg{\frac{p}{p_0}},
\end{equation}
\begin{explanation}
	где &$p_0$& пороговое звукового давление, $2\cdot{}10^{-5}\,\text{Па}/\text{м}^2$; \\
	&$p$& звуковое давление в данной точке, Па.
\end{explanation}

Понятие <<уровень звукового давления>> используется для измерения шума и для оценки его воздействия на человека, поскольку орган слуха чувствителен не к интенсивности, а к среднеквадратичному давлению~\cite{TechLib_NoiseCharacteristics}.

Некоторое представление об уровнях звукового давления даёт таблица~\ref{table:SoundLevels}.

\begin{table}[ht]
	\caption{Уровни звукового давления, создаваемые различными источниками шума}
	\def\arraystretch{1.5}
	\label{table:SoundLevels}
	\centering
	\begin{tabulary}{\textwidth}{|C|C|C|}
		\hline
		Источник шума & Расстояние от источника шума, м & Уровень звукового давления, дБ \\
		\hline
		Карманые часы & 1 & 20 \\
		\hline
		Шепот & 0,3 & 40 \\
		\hline
		Речь средней громкости & 1 & 60 \\
		\hline
		Металлорежущие станки & Рабочее место & 80--95 \\
		\hline
		Пневматическая клепка, обрубка & 1 & 110--115 \\
		\hline
		Поршневой авиационный двигатель & 2--3 & 120--130 \\
		\hline
		Реактивный двигатель & 2--3 (от выхлопа) & Свыше 140 \\
		\hline
	\end{tabulary}
\end{table}

При уровнях звукового давления около 140 дБ возникает физическая боль в ухе. Это так называемый <<болевой порог>> и дальнейшее повышение звукового давления может привести к разрыву барабанной перепонки~\cite{TechLib_NoiseCharacteristics}.

Логарифмическая шкала децибел даёт возможность определить только физическую характеристику шума, потому что она построена так, что пороговое значение звукового давления $p_0$ соответствует порогу слышимости на частоте 1000 Гц. В то же время слуховой аппарат человека неодинаково чувствителен к звукам различной частоты. Наибольшая чувствительность проявляется к звукам на средних и высоких частотах (от 800 до 4000 Гц), наименьшая~--- на низких (от 20 до 100 Гц). Вследствие этого для физиологической оценки шума приняты кривые равной громкости, полученные по результатам изучения свойств органа слуха оценивать звуки различной частоты по субъективному ощущению громкости, определяя, какой из них сильнее или слабее. Уровень громкости измеряется в фонах. На частоте 1000 Гц уровни громкости приняты равными уровням звукового давления. Зависимость среднеквадратичных значений синусоидальных составляющих шума (или соответствующих им уровней в дБ) от частоты называется частотным спектром шума (сокращённо спектром)~\cite{TechLib_NoiseCharacteristics}.

Спектры определяют, применяя анализаторы шума, которые имеют набор электрических фильтров, пропускающих сигнал в определенной полосе частот~--- полосе пропускания. Весь диапазон частот разбит на октавные полосы. Распространение получили приведённые ниже октавные полосы, в которых верхняя граничная частота в два раза больше нижней, а в качестве частоты, характеризующей полосу в целом, берётся среднегеометрическая частота (см. таблицу~\ref{table:OctaveBands})~\cite{TechLib_NoiseCharacteristics}.

\begin{table}[ht]
	\caption{Некоторые октавные полосы частот}
	\def\arraystretch{1.5}
	\label{table:OctaveBands}
	\centering
	\begin{tabulary}{\textwidth}{|C|C|}
		\hline
		Среднегеометрические частоты октавных полос, Гц & Граничные частоты октавных полос, Гц \\
		\hline
		63 & 45--90 \\
		\hline
		125 & 90--180 \\
		\hline
		250 & 180--355 \\
		\hline
		500 & 355--710 \\
		\hline
		1000 & 710--1400 \\
		\hline
		2000 & 1400--2800 \\
		\hline
		4000 & 2800--5600 \\
		\hline
		8000 & 5600--11200 \\
		\hline
	\end{tabulary}
\end{table}

% **************************************************
\subsection{Шум как вредный производственный фактор}
Технический прогресс во всех отраслях промышленности и на транспорте сопровождается разработкой и широким внедрением разнообразного оборудования, станков и транспортных средств. Рост мощностей современного оборудования и машин, развитие всех видов транспорта привели к тому, что человек на производстве и в быту постоянно подвергается воздействию шума высокой интенсивности~\cite{TechLib_NoiseProtection}.

Шум оказывает вредное влияние на весь организм и в первую очередь на центральную нервную и сердечно-сосудистую системы. Длительное воздействие интенсивного шума может привести к ухудшению слуха, а в отдельных случаях~--- к глухоте~\cite{TechLib_NoiseProtection}.

Шум на производстве неблагоприятно воздействует на работающего: ослабляет внимание, увеличивает расход энергии при одинаковой физической нагрузке, замедляет скорость психических реакций. В результате снижается производительность и ухудшается качество работы. Шум затрудняет также своевременную реакцию работающих на предупредительные сигналы, подаваемые персоналом, обслуживающим внутрицеховой транспорт (мостовые краны, автопогрузчики и т. п.), что может стать причиной несчастного случая~\cite{TechLib_NoiseProtection}.

% **************************************************
\subsection{Расчёт звукопоглощающей способности бетонной стены}
Исходные данные:
\begin{itemize}
	\item Шумное помещение~--- механический цех;
	\item Тихое помещение~--- кабинет мастера;
	\item Материал стены, разделяющей помещения~--- бетон, $\nu{} = 2400\,\text{кг}/\text{м}^3$;
	\item Толщина стены $d = 200\,\text{мм}$;
	\item Площадь стены, через которую проникает шум из цеха $S = 20\,\text{м}^2$;
	\item Объем помещения, изолируемого от шума $V = 150\,\text{м}^3$.
\end{itemize}

В ходе расчёта необходимо найти величину изоляции воздушного шума ограждающей конструкции для всех октавных полос частот, которые излучает источник шума:
\begin{equation}
	R_{\text{тр}} = L_{\text{ш}} - 10\lg{B_u} + 10\lg{S_i} - L_{\text{доп}} + 10\lg{n},
\end{equation}
\begin{explanation}
	где & $L_{\text{ш}}$ & октавный уровень звукового давления в шумном помещении, $\text{дБ}$; \\
	& $B_u$ & постоянная помещения, изолируемого от шума, $\text{м}^2$; \\
	& $S_i$ & площадь ограждающей конструкции (или её элемента), через которую шум проникает в помещение, $\text{м}^2$; \\
	& $L_{\text{доп}}$ & допустимый октавный уровень шума в защищаемом от шума помещении, дБ; \\
	& $n$ & общее количество ограждающих конструкций или их элементов, через которые проникает шум.
\end{explanation}

Постоянная $B_u$ определяется по формуле:
\begin{equation}
	B_u = B_{1000}\mu,
\end{equation}
\begin{explanation}
	где &$B_{1000}$& постоянная помещения в $\text{м}^2$, определяемая на среднегеометрической частоту 1000 Гц; \\
	&$\mu{}$& частотный множитель.
\end{explanation}

Звукоизолирующая способность однослойных ограждений с достаточной для практики точностью может быть определена по формуле:
\begin{equation}
	R = 20\lg{(mf)} - 47,5,
\end{equation}
\begin{explanation}
	где &$m$& поверхностная плотность ограждения, $\text{кг}/\text{м}^2$; \\
	&$f$& частота звука, Гц.
\end{explanation}
\begin{equation}
	m = \nu{}\cdot{}d,
\end{equation}
\begin{explanation}
	где &$\nu{}$& объёмный вес материала, $\text{кг}/\text{м}^3$; \\
	&$d$& толщина ограждающей конструкции, $\text{м}$;
\end{explanation}

В виду того, что в процессе расчётов необходимо произвести множество однотипных вычислений для разных частот, то для решения поставленной задачи целесообразно применить ПО для работы с электронными таблицами, а именно программу OpenOffice Calc из пакета OpenOffice. Полученные в результате расчётов данные занесём в таблицу~\ref{table:NoiseCalculations}.

\begin{table}[ht]
	\caption{Результаты расчёта звукоизолирующей способности стены и требуемых величин звукоизоляции для каждой из октавных полос частот}
	\def\arraystretch{1.5}
	\label{table:NoiseCalculations}
	\centering
	\begin{tabulary}{\textwidth}{|C|c|c|c|c|c|c|}
		\hline 
		Полоса частот, Гц & $L_{\text{ш}}$, дБ & $\mu$ & $B_u,\,\text{м}^2$ & $L_{\text{доп}}$, дБ & $R_{\text{тр}}$, дБ & $R$, дБ \\
		\hline 
		63   & 76 & 0,8  & 6     & 71 & 10 & 42 \\
		\hline 
		125  & 80 & 0,75 & 5,625 & 61 & 25 & 48 \\
		\hline 
		250  & 83 & 0,7  & 5,25  & 54 & 35 & 54 \\
		\hline 
		500  & 93 & 0,8  & 6     & 49 & 49 & 60 \\
		\hline 
		1000 & 95 & 1    & 7,5   & 45 & 54 & 66 \\
		\hline 
		2000 & 91 & 1,4  & 10,5  & 42 & 52 & 72 \\
		\hline
		4000 & 78 & 1,8  & 13,5  & 40 & 40 & 78 \\
		\hline
		8000 & 70 & 2,5  & 18,75 & 38 & 32 & 84 \\
		\hline
	\end{tabulary}
\end{table}

Таким образом, сравнив рассчитанные значения $R_{\text{тр}}$ И $R$ для каждой из октавных полос частот, можно сделать вывод о наличии надлежащей звукоизолирующей способности у рассматриваемой стены.