\sectioncentered*{Введение}
\addcontentsline{toc}{section}{Введение}
В современном мире борьба с акустическим шумом является одним из важнейших мероприятий по улучшению качества человеческой жизни. Так, например, автомобиль с низким уровнем шума является требованием рынка, что в свою очередь ставит перед производителем задачу поиска и устранения источников шума и улучшения качества звукоизоляции. Шум на производстве и вовсе является вредным фактором, с которым владелец предприятия обязан бороться. Всё это требует наличия эффективного способа для поиска источников шума. Один из таких способов предоставляет устройство под названием акустическая камера. Это устройство способно буквально показать место, где расположен источник шума. Создание акустической камеры стало возможно благодаря такой области знаний, как цифровая обработка сигналов, а благодаря последним достижениям в области вычислительной техники появилась возможность спроектировать компактное устройство, работающее в реальном времени.

Проектированию прототипа акустической камеры как раз и посвящён данный дипломный проект. К проектируемому устройству были предъявлены следующие требования:
\begin{itemize}
	\item Цена всех компонентов прототипа должна позволять разработать его в домашних условиях;
	\item Обработка данных должна производиться в реальном времени;
	\item Спектр регистрируемых прототипом звуковых волн должен быть как можно ближе к спектру человеческого голоса;
	\item Система должна быть масштабируемой (т. е. необходима возможность добавлять новые датчики или изменять их конфигурацию без изменения физических параметров вычислительного блока);
	\item Результаты обработки данных должны выводиться на цветной жидкокристаллический дисплей с разрешением не менее 1024 x 768 пикселей.
	\item В виду сложности обработки данных с двумерной микрофонной решётки и ограниченности времени отведённого на дипломный проект, в прототипе допустимо применить одномерную микрофонную решётку;
	\item Возможно отказаться от использования в проекте видеокамеры, т. к. она не является элементом, необходимым для демонстрации базовых принципов работы акустической камеры.
\end{itemize}

Для достижения поставленной цели необходимо будет решить ряд задач и выполнить ряд действий. А именно:
\begin{itemize}
	\item Изучить имеющуюся теоретическую информацию об акустических камерах и алгоритмах, которые могут быть применены при их проектировании (см. раздел~\ref{section:Theory});
	\item Сделать обзор технологий, которые которые будут применены в данном проекте (см. раздел~\ref{section:Technologies}).
	\item Рассчитать параметры микрофонной решётки (см. раздел~\ref{section:MicArrayCalculations});
	\item Разработать печатную плату, для соединения микрофонов с блоком обработки данных (см. раздел~\ref{section:PcbBuilding});
	\item Сформулировать требования к фильтру нижних частот для преобразования PDM-сигнала в PCM-формат, рассчитать коэффициенты этого фильтра и реализовать его на FPGA. При реализации следует учитывать преимущества и недостатки выбранной технологии, а также особенности решаемой разрабатываемым фильтром задачи для последующей его оптимизации (см. раздел~\ref{section:LowPassFilterBuilding});
	\item Построить бимформер с возможностью управления его диаграммой направленности (см. раздел~\ref{section:BeamformerBuilding});
	\item Разработать полосовой фильтр для снижения негативных эффектов от пространственного алиасинга и низкого пространственного разрешения на нижних частотах (см. раздел~\ref{section:BandPassFilterBuilding});  
	\item Разработать простейший видеоадаптер, осуществляющий вывод данных, полученных на предыдущих этапах, на ЖК монитор (см. раздел~\ref{section:VideoAdapterBuilding});
	\item Проанализировать результат работы построенной системы, описать проблемы, если таковые возникнут и предложить пути их решения (см. раздел~\ref{section:ResultAndProblems});
	\item Обосновать экономическую эффективность одного из применённых в проекте решений путём сравнения его себестоимости с себестоимостью альтернативного решения (см. раздел~\ref{section:Economy});
	\item Привести теоретические сведения о шуме и его воздействии на человека, чтобы показать актуальность устранения его источников, в том числе с использованием акустической камеры (см. раздел~\ref{section:OSH});
\end{itemize}