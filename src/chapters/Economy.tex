\section{Технико-экономическое обоснование}
\label{section:Economy}
Микрофонная решётка для прототипа акустической камеры, проектируемого в данном дипломном проекте может быть построена двумя способами: на основе цифровых PDM-микрофонов, либо на основе обыкновенных аналоговых микрофонов. Покажем экономическую эффективность первого способа, рассчитав стоимость каждого из решений.

% **************************************************
\subsection{Расчёт себестоимости решения на основе PDM-микрофонов}
Рассчитаем затраты на комплектующие и занесём полученные данные в таблицу~\ref{table:PdmMicsCost}.

\begin{table}[ht]
	\caption{Стоимость комплектующих при реализации микрофонной решётки на основе PDM-микрофонов}
	\def\arraystretch{1.5}
	\label{table:PdmMicsCost}
	\centering
	\begin{tabulary}{\textwidth}{|c|C|C|C|C|C|}
		\hline
		\textnumero~пп & Наименование работ, затрат & Единицы измерения & Количество & Цена, руб & Стоимость, руб \\
		\hline
		1 & Микрофон MP45DT02 & шт & 16 & 118 & 1888 \\
		\hline
		2 & Конденсатор 0805 & шт & 8 & 1 & 8 \\
		\hline
		3 & Провод МГТФ & м & 2 & 42 & 84 \\
		\hline
		4 & Припой ПОС-61, 100~г & часть & 0,05 & 767 & 38,35 \\
		\hline
		5 & Спиртоканифоль, 20~г & часть & 0,05 & 35 & 1,75 \\
		\hline
		6 & Стеклотекстолит односторонний & $\text{м}^2$ & 0,01 & 14600 & 146 \\
		\hline
		  & \textbf{ИТОГО} &&&& \textbf{2166,1} \\
		\hline
	\end{tabulary}
\end{table}

Рассчитаем транспортно-заготовительные расходы:
\begin{equation}
	\text{ЗАТР}_\text{тр.заг.} = 40\% \cdot{} \text{ЗАТР}_\text{компл.} = 0,4 \cdot{} 2166,1 = 866,44~\text{руб}.
\end{equation}

Вычислим итоговую величину материальных затрат:
\begin{equation}
	\text{ЗАТР}_\text{матер.} = \text{ЗАТР}_\text{тр.заг.} + \text{ЗАТР}_\text{компл.} = 3032,54~\text{руб}.
\end{equation}

Вычислим заработную плату по тарифу и занесём результаты в таблицу~\ref{table:PdmMicsSalary}.

\begin{table}[ht]
	\caption{Расчёт заработной платы по тарифу при реализации микрофонной решётки на основе PDM-микрофонов}
	\def\arraystretch{1.5}
	\label{table:PdmMicsSalary}
	\centering
	\begin{tabulary}{\textwidth}{|c|C|C|C|C|C|}
		\hline
		\textnumero~пп & Наименование работ, затрат & Единицы измерения & Количество & Цена, руб & Стоимость, руб \\
		\hline
		1 & Разработка технического задания & лист & 1 & 120 & 120 \\
		\hline
		2 & Разработка схемы соединений & лист & 1 & 540 & 540 \\
		\hline
		3 & Разработка печатной платы&  лист & 1 & 600 & 600 \\
		\hline
		4 & Разработка пояснительной записки & лист & 3 & 120 & 360 \\
		\hline
		5 & Резка печатных плат & шт & 4 & 20 & 80 \\
		\hline
		6 & Нанесение изображения печатных проводников & шт & 4 & 30 & 120 \\
		\hline
		7 & Травление печатных плат & шт & 4 & 25 & 100 \\
		\hline
		8 & Лужение печатных плат & шт & 4 & 35 & 140 \\
		\hline
		9 & Пайка пассивных компонентов & шт & 8 & 5 & 40 \\
		\hline
		10 & Пайка микрофонов & шт & 16 & 10 & 160 \\
		\hline
		11 & Отмывка флюса & шт & 4 & 15 & 60 \\
		\hline
		   & \textbf{ИТОГО} &&&& \textbf{2320} \\
		\hline
	\end{tabulary}
\end{table}

Рассчитаем премии:
\begin{equation}
	\text{ПРЕМИИ} = \text{ЗП}_\text{по тарифу} \cdot{} 30\% = 696~\text{руб}.
\end{equation}

Рассчитаем основную заработную плату:
\begin{equation}
	\text{ЗП}_\text{осн.} = \text{ЗП}_\text{по тарифу} + \text{ПРЕМИИ} = 3016~\text{руб}.
\end{equation}

Рассчитаем дополнительную заработную плату:
\begin{equation}
	\text{ЗП}_\text{доп.} = \text{ЗП}_\text{осн.} \cdot{} 10\% = 301,6~\text{руб}.
\end{equation}

Рассчитаем общий фонд заработной платы:
\begin{equation}
	\text{ЗП}_\text{общ.фонд} = \text{ЗП}_\text{осн.} + \text{ЗП}_\text{доп.} = 3317,6~\text{руб}.
\end{equation}

Отчисления на социальные нужды~--- обязательные отчисления по нормам, установленным законодательством органам государственного социального страхования от затрат на оплату труда работников, включаемых в себестоимость продукции (работ, услуг). Предприятия и организации производят отчисления на социальные нужды в размере 30 процентов от затрат на оплату труда, из которых:
\begin{itemize}
	\item 2,9\%~--- в Фонд социального страхования (ФСС);
	\item 5,1\%~--- в Федеральный фонд обязательного медицинского страхования (ФФОМС);
	\item 22\%~--- в Пенсионный фонд Российской Федерации.
\end{itemize}

Рассчитаем отчисления на социальные нужды:
\begin{equation}
	\text{РАСХ}_\text{соц.} = \text{ЗП}_\text{общ.фонд} \cdot{} 30\% = 995,28~\text{руб}.
\end{equation}

Рассчитаем накладные расходы:
\begin{equation}
	\text{РАСХ}_\text{накл.} = \text{ЗП}_\text{общ.фонд} \cdot{} 50\% = 1658,8~\text{руб}.
\end{equation}

Рассчитаем себестоимость рассматриваемого решения:
\begin{equation}
	\text{СС} = \text{ЗАТР}_\text{матер.} + \text{ЗП}_\text{общ.фонд} + \text{РАСХ}_\text{соц.} + \text{РАСХ}_\text{накл.} = 9004,22~\text{руб}.
\end{equation}

% **************************************************
\subsection{Расчёт себестоимости альтернативного решения на основе аналоговых микрофонов}
Рассчитаем затраты на комплектующие и занесём полученные данные в таблицу~\ref{table:AnalogMicsCost}.

\begin{table}[ht]
	\caption{Стоимость комплектующих при реализации микрофонной решётки на основе аналоговых микрофонов}
	\def\arraystretch{1.5}
	\label{table:AnalogMicsCost}
	\centering
	\begin{tabulary}{\textwidth}{|c|C|C|C|C|C|}
		\hline
		\textnumero~пп & Наименование работ, затрат & Единицы измерения & Количество & Цена, руб & Стоимость, руб \\
		\hline
		1 & АЦП PCM1870 & шт & 8 & 274 & 2192 \\
		\hline
		2 & Микрофон CMB-6544PF & шт & 16 & 94 & 1504 \\
		\hline
		3 & Резистор 0805 & шт & 16 & 1 & 16 \\
		\hline
		4 & Конденсатор 0805 & шт & 80 & 1 & 80 \\
		\hline
		5 & Припой ПОС-61, 100~г & часть & 0,07 & 767 & 53,69 \\
		\hline
		6 & Спиртоканифоль, 20~г & часть & 0,07 & 35 & 2,45 \\
		\hline
		7 & Стеклотекстолит односторонний & $\text{м}^2$ & 0,02 & 14600 & 292 \\
		\hline
		  & \textbf{ИТОГО} &&&& \textbf{4140,14} \\
		\hline
	\end{tabulary}
\end{table}

Рассчитаем транспортно-заготовительные расходы:
\begin{equation}
	\text{ЗАТР}_\text{тр.заг.} = 40\% \cdot{} \text{ЗАТР}_\text{компл.} = 0,4 \cdot{} 4140,14 = 1656,06~\text{руб}.
\end{equation}

Вычислим итоговую величину материальных затрат:
\begin{equation}
	\text{ЗАТР}_\text{матер.} = \text{ЗАТР}_\text{тр.заг.} + \text{ЗАТР}_\text{компл.} = 5796,2~\text{руб}.
\end{equation}

Вычислим заработную плату по тарифу и занесём результаты в таблицу~\ref{table:AnalogMicsSalary}.

\begin{table}[!ht]
	\caption{Расчёт заработной платы по тарифу при реализации микрофонной решётки на основе аналоговых микрофонов}
	\def\arraystretch{1.5}
	\label{table:AnalogMicsSalary}
	\centering
	\begin{tabulary}{\textwidth}{|c|C|C|C|C|C|}
		\hline
		\textnumero~пп & Наименование работ, затрат & Единицы измерения & Количество & Цена, руб & Стоимость, руб \\
		\hline
		1 & Разработка технического задания & лист & 1 & 210 & 210 \\
		\hline
		2 & Разработка схемы соединений & лист & 1 & 680 & 680 \\
		\hline
		3 & Разработка печатной платы & лист & 1 & 980 & 980 \\
		\hline
		4 & Разработка пояснительной записки & лист & 5 & 120 & 600 \\
		\hline
		5 & Резка печатных плат & шт & 8 & 20 & 160 \\
		\hline
		6 & Нанесение изображения печатных проводников & шт & 8 & 30 & 240 \\
		\hline
		7 & Травление печатных плат & шт & 8 & 25 & 200 \\
		\hline
		8 & Лужение печатных плат & шт & 8 & 35 & 280 \\
		\hline
		9 & Пайка пассивных компонентов & шт & 96 & 5 & 480 \\
		\hline
		10 & Пайка микрофонов & шт & 16 & 10 & 160 \\
		\hline
		11 & Пайка микросхем & шт & 8 & 10 & 80 \\
		\hline
		12 & Отмывка флюса & шт & 8 & 15 & 120 \\
		\hline
		   & \textbf{ИТОГО} &&&& \textbf{4190} \\
		\hline
	\end{tabulary}
\end{table}

Рассчитаем премии:
\begin{equation}
	\text{ПРЕМИИ} = \text{ЗП}_\text{по тарифу} \cdot{} 30\% = 1257~\text{руб}.
\end{equation}

Рассчитаем основную заработную плату:
\begin{equation}
	\text{ЗП}_\text{осн.} = \text{ЗП}_\text{по тарифу} + \text{ПРЕМИИ} = 5447~\text{руб}.
\end{equation}

Рассчитаем дополнительную заработную плату:
\begin{equation}
	\text{ЗП}_\text{доп.} = \text{ЗП}_\text{осн.} \cdot{} 10\% = 544,7~\text{руб}.
\end{equation}

Рассчитаем общий фонд заработной платы:
\begin{equation}
	\text{ЗП}_\text{общ.фонд} = \text{ЗП}_\text{осн.} + \text{ЗП}_\text{доп.} = 5991,7~\text{руб}.
\end{equation}

Рассчитаем отчисления на социальные нужды:
\begin{equation}
	\text{РАСХ}_\text{соц.} = \text{ЗП}_\text{общ.фонд} \cdot{} 30\% = 1797,51~\text{руб}.
\end{equation}

Рассчитаем накладные расходы:
\begin{equation}
	\text{РАСХ}_\text{накл.} = \text{ЗП}_\text{общ.фонд} \cdot{} 50\% = 2995,85~\text{руб}.
\end{equation}

Рассчитаем себестоимость рассматриваемого решения:
\begin{equation}
	\text{СС} = \text{ЗАТР}_\text{матер.} + \text{ЗП}_\text{общ.фонд} + \text{РАСХ}_\text{соц.} + \text{РАСХ}_\text{накл.} = 16581,26~\text{руб}.
\end{equation}

% **************************************************
\subsection{Вывод}
Результаты расчёта себестоимостей обоих решений говорят о том, что использование PDM-микрофонов вместо аналоговых позволяет снизить стоимость микрофонной решётки, состоящей из 16-ти микрофонов, с 16581 рубля 21 копейки до 9004 рублей 22 копеек, т. е. в 1,84 раза, что безусловно является хорошим показателем. Таким образом, использованное в данной дипломной работе решения на основе PDM-микрофонов является экономически обоснованным.