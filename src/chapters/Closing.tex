\sectioncentered*{Заключение}
\addcontentsline{toc}{section}{Заключение}
В данной дипломной работе был выполнен ряд действий и решён ряд задач, таких как:
\begin{itemize}
	\item Изучена имеющаяся теоретическая информацию об акустических камерах;
	\item Сделан обзор технологий, которые которые в последствии были применены в данном проекте;
	\item Рассчитаны параметры микрофонной решётки;
	\item Разработана печатная плата, для соединения микрофонов с блоком обработки данных;
	\item Сформулированы требования и осуществлено проектирование 16-ти канального фильтра нижних частот 4095-го порядка для преобразования PDM-сигнала в PCM-формат;
	\item Построен бимформер с возможностью управления его диаграммой направленности;
	\item Разработан полосовой фильтр с бесконечной импульсной характеристикой;
	\item Разработан простейший видеоадаптер, осуществляющий вывод обработанных данных;
	\item Проанализирован результат работы полученной системы, описаны её проблемы и недостатки, а также выдвинут и проверен ряд гипотез для их устранения.
	\item Обоснована экономическая эффективность применения цифровых PDM-микрофонов вместо аналоговых.
	\item Приведены теоретические сведения о шуме и его воздействии на человека.
\end{itemize}

Таким образом цель работы была достигнута. Однако, остаётся ряд вопросов, вынесенных за пределы данной работы. Одним из таких вопросов является улучшение пространственной чувствительности разработанного устройства. Этот вопрос требует проведения дополнительных экспериментов. Также за пределами работы оставлены задачи обработки данных с двумерной решётки и наложения звуковой картины на видеоизображение. Без решения этих задач невозможно построение устройства, которое в последствии можно будет выпустить на рынок.